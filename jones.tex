\subsection{Rodney Jones}
\subsubsection{Introduction}
I am provisional PhD Security candidate the University of Colorado Colorado Springs.One of my goals for the course is to improve my research skills. I have an interest in politics and, of course, computers. I have spent 24 years in the Army Reserves. I was also trained in the Army as a Russian linguist/interrogator. Although I never had the opportunity or the capabilities to interrogate anyone in Russian, I have always been fascinated with Russian culture.My research will be in the area of Russian disinformation directed toward African Americans on social media. As an African American, I am aware of the racial bias toward other races in Russia and other European countries.Although we have the capability to communicate with almost anyone anywhere due to advances in technology, some of our discourse as of late has been affected by disinformation. \\


\subsection{My Picture}
\subsubsection{My Picture}
\begin{figure}[htp]
    \centering
    \includegraphics[scale=.095]{jones-pic}
    \caption{Rodney Jones}
 \end{figure}

\subsection{Related GiT repository}
\subsubsection{Related GiT repository}
https://github.com/DFRLab/Dichotomies-of-Disinformation/blob/master/README.md \\

\subsection{Q \& A}
\subsubsection{Q \& A}
    \begin{enumerate}
How to determine when or if misinformation becomes disinformation?
Can test be sporadically conducted to test users capability to identify false information?
Can user be allowed to label posts with options such as verified, unverified, or given the option to correct a post previously sent?
